\section{Introduction}

Since the beginning of recorded time we have wondered whether we are
alone or whether there are other planets that might support life. The
scientific quest for extrasolar or exoplanets (planets beyond our own
solar system) began in the mid 19th century, but it was only as recent
as 1992 that astonomers have been able to confirm the existence of
other planetary systems.  Since then {\tt describe resources,
  telescopes, etc} have been devoted to identifying stars with
orbiting planets, and in particular, those that might have earth-like
planets. As of June 2010 over 450 exoplanets have been detected, the
majority using techniques that rely on measurements of a star's radial
velocity The presence of a planet orbiting a star will lead to a
periodic wobble in the the star's radial velocity, while if there are
no planets present the expected velocity is constant although
measurements fluctuate because of stellar jitter and {\tt other
  explanations for noise}.  Based on a short sequence of noisy
measurements, the challenge is to detemine whether there are no
planets or one, two or more planets, and if there are planets to
characterize their orbital parameters.  {\tt cite work of Gregory and
  Ford, but mention limitations?}  If the data do not provide strong
evidence either in favor or against the presence of planets,
additional observations may be needed to provide confirmation.
Determining which stars to observe and when to schedule scarce
telescope time optimally is critical given limited resources; Bayesian methods in particular
are ideally suited for such sequential updating of
information/evidence and decision making in such sampling problems.


The primary question of whether there are no planets or one or more
may be poised as a Bayesian model selection problem, where the
collection of models under consideration are the zero-planet mode
$\M_0$, one-planet $\M_1$, two-planet $\M_2$, etc.  In Bayesian
statistics, the posterior probability of a model $\M_j$ requires
accurately calculating the integrated or marginal likelihood and in
the setting of exo-planet is extremely challenging; see
\citet{Clyd:Geor:2004} for a general overview or \cite{Clyd:etal:2005}
for some of the issues that arise in astronomy. Out of available
methods, importance sampling (IS) generally produces the best
estimates of marginal likelihoods in terms of bias and variance. The
key to building an efficient IS algorithm is to design an importance
function that mimics the integrand.  Building such a function can be
quite challenging even in lower to modest dimensional settings, with
multi-modal posterior distributions that arise in the exo-planet
setting the task is all the more difficult. 

In this paper, we propose a novel adaptive method for constructing an
importance function using a mixture of Student-$t$ distributions that
evolves to the target posterior distribution. We present in Section
\ref{sec:exo} an overview of the scientific issues that inspired this
work, namely inferring the number of planets around distant stars. In
the exoplanet application, each model $\M_p$ has $2+5p$ parameters
where $p$ is the number of planets in the model, so even the simplest
single planet model has a seven dimensional parameter space to
integrate over. While low dimensional compared to many modern problems
of model selection, the models are highly nonlinear making Laplace
approximations untenable given the modest sample sizes and multi-modal
posterior distribution; even sampling from posterior distributions
using Markov chain Monte Carlo methods is in itself a challenging
problem.  In Section \ref{sec:IS} we review importance sampling and
how it may be used to calculate marginal likelihoods for model
selection. In Section \ref{sec:AAIS} we describe the Adaptive Annealed
Importance Sampling method developed for the exoplanet problem. This
method integrates several key features of other methods: mixture
models, annealling, and adaptation via a sequence of distributions to
build a flexible importance distribution that mimics the posterior
distribution, but incorporates ideas successfully used in
trans-dimensional methods to adapt the number of kernels in the
mixture by utilizing split and merge moves.  In Section
\ref{sec:simulation} we present two simulation studies that
demonstrate the efficiency of the method compared to other approaches.
The first involves a challenging flared helix in three dimensions,
while the second seven dimensional problem that was selected to reflex
the target function in the exoplanet problem.  In Section
\ref{sec:app} we illustrate the method in the analysis of
several real star systems with one and two planets. Finally, we conclude with
discussion of possible extensions of the method in Section
\ref{sec:disc}.
